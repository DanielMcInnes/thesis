\documentclass[12pt,openany,a4paper]{book}
\usepackage{graphics}	% if you want encapsulated PS figures.
\usepackage{graphicx}
\usepackage[colorlinks,allcolors=blue]{hyperref, xcolor} % must be bfore pgfplots
\usepackage{pgfplots} % must be after xcolor
\usepackage{colortbl}
% If you use a macro file called macros.tex :
% \input{macros}
% Note: The present document has its macros built in.

% Number subsections but not subsubsections:
\setcounter{secnumdepth}{2}
% Show subsections but not subsubsections in table of contents:
\setcounter{tocdepth}{2}

\pagestyle{headings}		% Chapter on left page, Section on right.
\raggedbottom

\setlength{\topmargin}		{-5mm}  %  25-5 = 20mm7
\setlength{\oddsidemargin}	{10mm}  % rhs page inner margin = 25+10mm
\setlength{\evensidemargin}	{0mm}   % lhs page outer margin = 25mm
\setlength{\textwidth}		{150mm} % 35 + 150 + 25 = 210mm
\setlength{\textheight}		{240mm} % 

\renewcommand{\baselinestretch}{1.2}	% Looks like 1.5 spacing.

% Stop figure/tables smaller than 3/4 page from appearing alone on a page:
\renewcommand{\textfraction}{0.25}
\renewcommand{\topfraction}{0.75}
\renewcommand{\bottomfraction}{0.75}
\renewcommand{\floatpagefraction}{0.75}

% THEOREM-LIKE ENVIRONMENTS:
\newtheorem{defn}	{Definition}	% cf. \dfn for cross-referencing
\newtheorem{theorem}	{Theorem}	% cf. \thrm for cross-referencing
\newtheorem{lemma}	{Lemma}		% cf. \lem for cross-referencing

% AIDS TO CROSS-REFERENCING (All take a label as argument):
\newcommand{\eref}[1] {(\ref{#1})}		% (...)
\newcommand{\eq}[1]   {Eq.\,(\ref{#1})}		% Eq.~(...)
\newcommand{\eqs}[2]  {Eqs.~(\ref{#1}) and~(\ref{#2})}
\newcommand{\dfn}[1]  {Definition~\ref{#1}}	% Definition~...
\newcommand{\thrm}[1] {Theorem~\ref{#1}}	% Theorem~...
\newcommand{\lem}[1]  {Lemma~\ref{#1}}		% Lemma~...
\newcommand{\fig}[1]  {Fig.\,\ref{#1}}		% Fig.~...
\newcommand{\tab}[1]  {Table~\ref{#1}}		% Table~...
\newcommand{\chap}[1] {Chapter~\ref{#1}}	% Chapter~...
\newcommand{\secn}[1] {Section~\ref{#1}}	% Section~...
\newcommand{\ssec}[1] {Subsection~\ref{#1}}	% Subsection~...

% AIDS TO FORMATTING:
\newcommand{\teq}[1]	{\mbox{$#1$}}	% in-Text EQuation (unbreakable)
\newcommand{\qed}	{\hspace*{\fill}$\bullet$}	% end of proof

% MATHEMATICAL TEMPLATES:
% Text or math mode:
\newcommand{\half}	{\ensuremath{\frac{1}{2}}}	% one-half
\newcommand{\halftxt}	{\mbox{$\frac{1}{2}$}}	  	% one-half, small
% Math mode only:
% N.B. Parentheses are ROUND; brackets are SQUARE!
\newcommand{\oneon}[1]	{\frac{1}{#1}}		  % reciprocal
\newcommand{\pow}[2]	{\left({#1}\right)^{#2}}  % Parenthesized pOWer
\newcommand{\bow}[2]	{\left[{#1}\right]^{#2}}  % Bracketed pOWer
\newcommand{\evalat}[2]	{\left.{#1}\right|_{#2}}  % EVALuated AT with bar
\newcommand{\bevalat}[2]{\left[{#1}\right]_{#2}}  % Bracketed EVALuated AT
% Total derivatives:
\newcommand{\sdd}[2]	{\frac{d{#1}}{d{#2}}}		    % Short
\newcommand{\sqdd}[2]	{\frac{d^2{#1}}{d{#2}^2}}	    % 2nd ("SQuared")
\newcommand{\ldd}[2]	{\frac{d}{d{#1}}\left({#2}\right)}  % Long paren'ed
\newcommand{\bdd}[2]	{\frac{d}{d{#2}}\left[{#2}\right]}  % long Bracketed
% Partial derivatives (same sequence as for total derivatives):
\newcommand{\sdada}[2]	{\frac{\partial {#1}}{\partial {#2}}}
\newcommand{\sqdada}[2]	{\frac{\partial ^{2}{#1}}{\partial {#2}^{2}}}
\newcommand{\ldada}[2]	{\frac{\partial}{\partial {#1}}\left({#2}\right)}
\newcommand{\bdada}[2]	{\frac{\partial}{\partial {#1}}\left[{#2}\right]}
\newcommand{\da}	{\partial}

% ORDINAL NUMBERS:
\newcommand{\ith}	{\ensuremath{i^{\rm th}}}
\newcommand{\jth}	{\ensuremath{j^{\rm th}}}
\newcommand{\kth}	{\ensuremath{k^{\rm th}}}
\newcommand{\lth}	{\ensuremath{l^{\rm th}}}
\newcommand{\mth}	{\ensuremath{m^{\rm th}}}
\newcommand{\nth}	{\ensuremath{n^{\rm th}}}

% SINUSOIDAL TIME AND SPACE-DEPENDENCY FACTORS:
\newcommand{\ejot}	{\ensuremath{e^{j\omega t}}}
\newcommand{\emjot}	{\ensuremath{e^{-j\omega t}}}

% UNITS (TEXT OR MATH MODE, WITH LEADING PADDING SPACE IF APPLICABLE):
% NB: These have not been tested since being modified for LaTeX2e.
\newcommand{\pack}	{\hspace{-0.08em}}
\newcommand{\Pack}	{\hspace{-0.12em}}
\newcommand{\mA}	{\ensuremath{\rm\,m\pack A}}
\newcommand{\dB}	{\ensuremath{\rm\,d\pack B}}
\newcommand{\dBm}	{\ensuremath{\rm\,d\pack B\pack m}}
\newcommand{\dBW}	{\ensuremath{\rm\,d\pack B\Pack W}}
\newcommand{\uF}	{\ensuremath{\rm\,\mu\pack F}}
\newcommand{\pF}	{\ensuremath{\rm\,p\pack F}}
\newcommand{\nF}	{\ensuremath{\rm\,n\pack F}}
\newcommand{\uH}	{\ensuremath{\rm\,\mu\pack H}}
\newcommand{\mH}	{\ensuremath{\rm\,m\pack H}}
\newcommand{\Hz}	{\ensuremath{\rm\,H\pack z}}
\newcommand{\kHz}	{\ensuremath{\rm\,k\pack H\pack z}}
\newcommand{\MHz}	{\ensuremath{\rm\,M\pack H\pack z}}
\newcommand{\GHz}	{\ensuremath{\rm\,G\pack H\pack z}}
\newcommand{\J}		{\ensuremath{\rm\,J}}
\newcommand{\kg}	{\ensuremath{\rm\,k\pack g}}
\newcommand{\K}		{\ensuremath{\rm\,K}}
\newcommand{\m}		{\ensuremath{\rm\,m}}
\newcommand{\cm}	{\ensuremath{\rm\,cm}}
\newcommand{\km}	{\ensuremath{\rm\,k\pack m}}
\newcommand{\mm}	{\ensuremath{\rm\,m\pack m}}
\newcommand{\nm}	{\ensuremath{\rm\,n\pack m}}
\newcommand{\um}	{\ensuremath{\rm\,\mu m}}
\newcommand{\Np}	{\ensuremath{\rm\,N\pack p}}
\newcommand{\s}		{\ensuremath{\rm\,s}}
\newcommand{\ms}	{\ensuremath{\rm\,m\pack s}}
\newcommand{\us}	{\ensuremath{\rm\,\mu s}}
\newcommand{\V}		{\ensuremath{\rm\,V}}
\newcommand{\mV}	{\ensuremath{\rm\,m\Pack V}}
\newcommand{\W}		{\ensuremath{\rm\,W}}
\newcommand{\mW}	{\ensuremath{\rm\,m\Pack W}}
\newcommand{\ohm}	{\ensuremath{\rm\,\Omega}}
\newcommand{\kohm}	{\ensuremath{\rm\,k\Omega}}
\newcommand{\Mohm}	{\ensuremath{\rm\,M\Omega}}
\newcommand{\degs}	{\ensuremath{\rm^{\circ}}}

% LaTeX run-time type-in command:
%
% \typein{Enter \protect\includeonly{...} command (or just type RETURN):}
%
% Uncommenting this command makes LaTeX prompt you for the \includeonly
% list.  At the prompt
%
%	\@typein=
%
% you type
%
%	\includeonly{chap1,chap2}
%
% to include the files chap1.tex and chap2.tex and omit any others.
% To include every \include file, just hit RETURN.
% If you are running LaTeX from xtexsh, you may need to click the mouse
% in the LaTeX window to position the cursor at the \@typein prompt.
\graphicspath{{/home/dmcinnes/Dropbox/thesis/thesis/images/}}
\begin{document}

\frontmatter
% By default, frontmatter has Roman page-numbering (i,ii,...).

\begin{titlepage}
\renewcommand{\baselinestretch}{1.0}
\begin{center}
\includegraphics{UQlogo}\\
\vspace*{35mm}
\Huge\bf
		A VERIFIED OCPP v2.0 SERVER\\
		FOR ELECTRIC VEHICLE CHARGER NETWORKS\\
\vspace{20mm}
\large\sl
		by\\
		DANIEL HUGH MCINNES
		\medskip\\
\rm
		School of Information Technology and Electrical Engineering,\\
		The University of Queensland.\\
\vspace{30mm}
		Submitted for the degree of\\
		Master of Engineering
		\smallskip\\
\normalsize
		in the field of Software Engineering
		\medskip\\
\large
		October 2019.		
\end{center}
\end{titlepage}

\cleardoublepage

\begin{flushright}
	Daniel McInnes\\
	s4231125@student.uq.edu.au\\
	\medskip
	\today
\end{flushright}
\begin{flushleft}
  Prof Amin Abbosh\\
  Acting Head of School\\
  School of Information Technology and Electrical Engineering\\
  The University of Queensland\\
  St Lucia, Q 4072\\
  \bigskip\bigskip
  Dear Professor Abbosh,
\end{flushleft}

In accordance with the requirements of the degree of Master of
Engineering in the division of 
Software Engineering,
I present the
following thesis entitled ``A Verified Server for an Electric Vehicle Charger Network''.  This work was performed under the supervision of A/Prof. Graeme Smith.

I declare that the work submitted in this thesis is my own, except as
acknowledged in the text and footnotes, and has not been previously
submitted for a degree at The University of Queensland or any other
institution.

\begin{flushright}
	Yours sincerely,\\
	\medskip
%	\emph{Author's Signature}\\
	\medskip
	Daniel McInnes.
\end{flushright}

\cleardoublepage


\chapter{Acknowledgments}

I wish to acknowledge the support of my supervisor, Associate Professor Graeme Smith, whose expertise and assistance were greatly appreciated.
\cleardoublepage

\chapter{Abstract}

% Notice that all \include files are chapters -- a logical division.
% But not all chapters are \include files; some chapters are short
% enough to be in-lined in the main file.
Currently, networks of publicly available electric vehicle fast chargers communicate with servers using the Open Charge Point Protocol (OCPP). Ideally, the software running on these servers would be error free and never crash. Numerous software verification tools exist to prove desirable properties of the server software, such as functional correctness, the absence of race conditions, memory leaks, and certain runtime errors. I compare the different features of several verification tools, and use one of them to partially implement an OCPP server.

\tableofcontents

\listoffigures
\addcontentsline{toc}{chapter}{List of Figures}

\listoftables
\addcontentsline{toc}{chapter}{List of Tables}

% If file los.tex begins with ``\chapter{List of Symbols}'':
% \include{los}

\cleardoublepage

\mainmatter
% By default, mainmatter has Arabic page-numbering (1,2,...).


% Chapters may be \include files, each beginning with a line like
%
%	\chapter{Title of chapter}
%
% e.g. if two chapter files were called intro.tex and theory.tex,
% we would say
%
%	\include{intro}
%	\include{theory}

\chapter{Introduction}



The International Energy Agency \cite{ieaglobalevoutlook} reports that the number of publicly available fast chargers ($>$ 22kW) increased from 107\,650 in 2017 to 143\,502 in 2018.\\

\begin{figure}[htbp]
\caption{Publicly available fast chargers}


\begin{tikzpicture}
	\begin{axis}
	[
		ybar,
		scaled y ticks = false,
      		y tick label style={/pgf/number format/fixed,/pgf/number format/1000 sep = \thinspace},
		symbolic x coords={2007, 2008, 2009, 2010, 2011, 2012, 2013, 2014, 2015, 2016, 2017, 2018},
   	 	y label style={at={(axis description cs:-0.1,.5)},anchor=south},
		ylabel={Number of chargers},
		x label style={at={(axis description cs:0.5,-0.2)},anchor=north},
		xlabel={Year},
		x tick label style = {font = \small, text width = 1.7cm, align = center, rotate = 70, anchor = north east},
		xtick=data
	]
    	\addplot[ybar,fill=blue] coordinates {
        		(2007,42)
        		(2008,42)
        		(2009,47)
		(2010,372)
        		(2011,1356)
        		(2012,3332)
        		(2013,5044)
        		(2014,16762)
        		(2015,26784)
        		(2016,73851)
        		(2017,107650)
        		(2018,143502)
	};
\end{axis}
\end{tikzpicture}
\end{figure}


The Open Charge Alliance \cite{ocaappraisal} reports that more than 10\,000 charging stations\footnote{Note that a ``charging station'' may consist of multiple fast chargers, thus the disparity between 143\,502 ``fast chargers'' and ``more than 10\,000 charging stations''} in over 50 countries are managed using the OCPP protocol.


 
In this paper I investigate and compare the features of several different software verifiation tools and choose one to partially implement an OCPP server. The tools include VeriFast, VCC, OpenJML, KeY, Spec\#, Daphney, Whiley, and AdaSPARK.

The tools vary in what guarantees they provide. Desirable guarantees that I was interested in are: 
\label{criteria}
\begin{itemize}
	\item the absence of memory leaks
	\item the program should never access uninitialized memory (for example, should never read past the end of an array)
	\item the program should never crash or exit unexpectedly
	\item the tool should verify the absense of stack overflows, i.e. the program should be bounded in terms of memory (RAM) usage at runtime
	\item the program should verifiably meet its requirements. These requirements are typically in the form of preconditions and postconditions.
	\item the program should never exhibit undefined behaviour
	\item the program should constrain information flow, i.e. not leak sensitive information such as passwords
	\item The tool should be sound. Many of the tools claim to be sound ``modulo bugs in the tool'', and have lengthy lists of known bugs. I want a tool that has no known unsound behaviour.

\end{itemize}

\chapter{Literature review / prior art}

------------------------------------------------------------------
Background (20\%): Background material for the thesis should likely include reviews, analyses and
discussions of the literature in the area of the thesis and about methods applicable to achieving
the thesis goals. This background should not only help the reader understand the rest of the document,
but should illustrate to the reader a clear mastery of the material in the topic area and an
ability to synthesize and abstract knowledge from other sources.

You will need to review previous work in the field, which may include books and
papers (“literature”), patents and commercial products (“prior art”), and earlier
work in your Department. This information is usually (but not always) collected in
a single chapter, whose title should preferably be more specific and interesting than
the one above.

------------------------------------------------------------------

10 different verification tools were evaluated to determine which fulfilled the criteria \ref{criteria}.

\section {Dafny}



\section {JML}
\section {KeY}
\section {SPARK Ada}
\section {Spec\#}
\section {Verifast}
Vogels claims VeriFast is to be sound. However, 

Verifast guarantees (modulo bugs in the tool) the absence of illegal memory accesses such as null pointer dereferences, accesses of uninitialized memory, accessing outside of the bounds of an array, data race conditions, violations of the user specified function contracts.
VeriFast performs modular formal verification ofcorrectness properties of single and multithreaded C and Java programs. \\

%(Vogels et al, p1. ``VeriFast is a sound modular formal verification approach for single-threaded and multithreaded imperative programs being developed at KU Leuven''.) 
\section {VCC}
\section {Viper}
\section {Whiley}
\section {Why3}
``Why3 is a platform for deductive program verification. It provides a rich language for
specification and programming, called WhyML, and relies on external theorem provers,
both automated and interactive, to discharge verification conditions.''\\
``A user can write WhyML programs directly and get correct-by-construction
OCaml programs through an automated extraction mechanism. WhyML is also used as
an intermediate language for the verification of C, Java, or Ada programs.''



You will need to review previous work in the field, which may include
books and papers (``literature''), patents and commercial products
(``prior art''), and earlier work in your Department.  This
information is usually (but not always) collected in a single chapter,
whose title should preferably be more specific and interesting than
the one above.

\section{Summary of Verification Tool Features}

\begin{tabular}{ |p{1.3cm}||p{1.5cm}|p{1.5cm}|p{1.5cm}| p{1.5cm}|p{1.5cm}|p{1.5cm}|p{1cm}|p{1cm}| p{1cm}| }
 \hline
 \multicolumn{9}{|c|}{Features} \\
 \hline
Tool & 
No memory leaks & 
Never accesses uninitialized memory & 
Never crashes / exits unexpectedly &
Bounded
RAM &
Never hang &
Prove correct &
No undefined &
No data leak 
\\
 \hline
Dafny 		& \cellcolor{red} \ref{unproven} & \cellcolor{red} \ref{unproven} & \cellcolor{red} \ref{dafnybug1} & \cellcolor{red} & \cellcolor{green} & \cellcolor{green} & ? & \cellcolor{red} \ref{dafnybug2} \\
 \hline
JML   		& ? & ? & ?& ?& ?& ?& ?& ?\\
 \hline
KeY 		& ? & ? & ?& ?& ?& ?& ?& ?\\
 \hline
SPARK		& ? & ? & ?& ?& ?& ?& ?& ?\\
 \hline
Spec \#		& ? & ? & ?& ?& ?& ?& ?& ?\\
 \hline
Verifast& ? & ? & ?& ?& ?& ?& ?& ?\\
 \hline
VCC& ? & ? & ?& ?& ?& ?& ?& ?\\
 \hline
Viper& ? & ? & ?& ?& ?& ?& ?& ?\\
 \hline
Whiley& ? & ? & ?& ?& ?& ?& ?& ?\\
 \hline
Why3& ? & ? & ?& ?& ?& ?& ?& ?\\
 \hline
\end{tabular}

\label{unproven} Internet searches failed to find evidence that this is supported.

\chapter{Theory}

A scientific paper is likely to be read by people who are not
specialists in the same field as the author(s), but who nevertheless
may need to use the results of the paper in their own fields.
Similarly, the examiners of your thesis will probably include at least
one academic who does not teach or conduct research in the subject
area of your thesis.  In an early chapter of your thesis, therefore,
you should quote any theoretical results which are necessary for the
understanding of later chapters.  Examiners who are not specialists in
your area will know whether you have given sufficient theoretical
information.  They will also know whether you have insulted their
status by presenting material which is familiar to every
half-competent graduate in every field of ECE.

\chapter{Methodology, procedure, design, etc.}

This may be one chapter or several.  Again, titles should be more
informative than the above.

You will almost certainly need diagrams to clarify your meaning.  The
\LaTeXe\ \texttt{graphics} package allows the inclusion of PostScript
graphics, as in .  The inclusion of \LaTeX\ \texttt{picture}
graphics, as in , requires no auxiliary packages and allows
the mathematical formatting features of \LaTeX\ to be used in
diagrams; but the \texttt{picture} files, unlike PostScript files,
usually require manual editing.


\chapter{Results and discussion \ldots}

\ldots\ or perhaps the discussion should be a separate chapter.

In any case, you will probably need to include tabulated results.
\tab{tf2} illustrates the use of various \LaTeX\ environments to
include a computer printout (plain text file) in a document.  The
\texttt{verbatim} environment, which encloses the formatted text, is
also useful for program listings.

\chapter{Conclusions}

\section{Summary and conclusions}

\section{Possible future work}

\appendix

% Chapters after the \appendix command are lettered, not numbered.
% Setting apart the appendices in the table of contents is awkward:

\newpage
\addcontentsline{toc}{part}{Appendices}
\mbox{}
\newpage

% The \mbox{} command between two \newpage commands gives a blank page.
% In the contents, the ``Appendices'' heading is shown as being on this
% blank page, which is the page before the first appendix.  This stops the
% first appendix from be listed ABOVE the word ``Appendices'' in the
% table of contents.

% \include appendix chapters here.

\chapter{Dummy appendix}

\section{Dafny}


\label {dafnybug1}

\begin{verbatim}

https://github.com/dafny-lang/dafny/issues/532
Simulated type set crashes at run-time \#532
An attempt to do a dynamic type test causes a crash when the compiled program is run. This should either be disallowed statically or should compile to good code.

Repro: Here is the output on the program below:

\$ dafny /compile:3 test.dfy
Dafny 2.3.0.10506
test.dfy(17,19): Warning: /!\ No terms found to trigger on.

Dafny program verifier finished with 2 verified, 0 errors
Running...

t.x=5  The given Tr is a C, and c.y=6
t.x=100  Error: Execution resulted in exception: Exception has been thrown by the target of an invocation.
System.InvalidCastException: Specified cast is not valid.
%  at _module.__default+<M>c__AnonStorey0.<>m__0 () [0x00023] in <73b9bd36ee6e47fba3617ec618048be5>:0
%  at _module.__default.M (_module.Tr t) [0x0003b] in <73b9bd36ee6e47fba3617ec618048be5>:0
%  at _module.__default.Main () [0x00058] in <73b9bd36ee6e47fba3617ec618048be5>:0
%  at (wrapper managed-to-native) System.Reflection.MonoMethod.InternalInvoke(System.Reflection.MonoMethod,object,object[],System.Exception&)
  at System.Reflection.MonoMethod.Invoke (System.Object obj, System.Reflection.BindingFlags invokeAttr, System.Reflection.Binder binder, System.Object[] parameters, System.Globalization.CultureInfo culture) [0x00032] in <bb7b695b8c6246b3ac1646577aea7650>:0



And here is the program:

trait Tr {
  var x: int
}

class C extends Tr {
  var y: int
}

class D extends Tr {
  var z: int
}

method M(t: Tr)
  modifies t
{
  print "t.x=", t.x, "  ";
  var s: set<C> := set c: C | c == t;  // this line crashes for the call M(d)
  if s == {} {
%    print "The given Tr is not a C\n";
  } else {
    var c :| c in s;
%    print "The given Tr is a C, and c.y=", c.y, "\n";
    c.y := c.y + 10;
  }
}

method Main() {
  var c := new C;
  var d := new D;
  c.x, c.y := 5, 6;
  d.x, d.z := 100, 102;

  M(c);
  M(d);
  M(c);
}

\end{verbatim}

\label{dafnybug2}
\begin{verbatim}
https://gitter.im/dafny-lang/community?at=5d90c402086a72719e848f24
Bryan Parno
@parno
Mar 14 05:41
We use reference counting via shared_ptr Which means it is possible to create memory leaks if you try hard enough (e.g., via a doubly-linked list). For the latter case, I've been thinking about adding an annotation that a developer could add that would give the compiler a hint that it should use a weak_ptr to break up such cycles, but I haven't gotten around to it yet
\end{verbatim}





Appendices are useful for supplying necessary details or explanations
which do not seem to fit into the main text, perhaps because they are
too long and would distract the reader from the central argument.
Appendices are also used for program listings.

Notice that appendices are ``numbered'' with capital letters, not
numerals.  When the \verb+\appendix+ command in
\LaTeX~\cite[p.\,175]{lamport} is used with the \texttt{book} document
class, it causes subsequent chapters to be treated as appendices.

\chapter{Program listings}

\section{First program}

Some initial explanatory notes may precede the listing.

\section{Second program}

\section{Etc.}

\chapter{Companion disk}

See https://github.com/DanielMcInnes/thesis.git .

\cleardoublepage

\begin{thebibliography}{99}
\addcontentsline{toc}{chapter}{Annotated Bibliography}


% IEA Global EV Outlook 2019
\bibitem{ieaglobalevoutlook} International Energy Agency, ``Global EV Outlook 2019'', \textit{International Energy Agency}, May 2019. Available: \href{https://webstore.iea.org/global-ev-outlook-2019}{\color{blue}{\underline{https://webstore.iea.org/global-ev-outlook-2019}}}. [Accessed June 25, 2019]. \\
2. \\
3. \\
4. \\
5.  \\
6. \\
7. \\
8. \\



% https://www.openchargealliance.org/about-us/appraisal-ocpp/
\bibitem{ocaappraisal} Open Charge Alliance, ``Appraisal OCPP'',  todo - undated. [online]. Available: \href{https://www.openchargealliance.org/about-us/appraisal-ocpp/}{\color{blue}{\underline{https://www.openchargealliance.org/about-us/appraisal-ocpp/}}}. [Accessed June 22, 2019]. \\
3. \\
4. \\
5. \\
6. \\
7. \\
8. \\

% https://www.openchargealliance.org/protocols/ocpp-20/
\bibitem{ocalwebsite} Open Charge Alliance, ``Open Charge Point Protocol 2.0'', todo - undated. [online]. Available: \href{https://www.openchargealliance.org/protocols/ocpp-20/}{\color{blue}{\underline{https://www.openchargealliance.org/protocols/ocpp-20/}}}. [Accessed June 22, 2019]. \\
3. \\
4. \\
5. \\
6. \\
7. \\
8. \\



% https://github.com/verifast/verifast
\bibitem{verifastwebsite} VeriFast, ``Research prototype tool for modular formal verification of C and Java programs'', todo - undated. [online]. Available: \href{https://github.com/verifast/verifast/}{\color{blue}{\underline{https://github.com/verifast/verifast/}}}. [Accessed June 22, 2019]. \\
2. This is the website for the VeriFast verification tool.\\
3. \\
4. \\
5. \\
6. \\
7. \\
8. \\



\bibitem{featherweightverifast} F. Vogels, B. Jacobs, and F. Piessens, ``Featherweight Verifast'', \emph{Logical Methods in Computer Science}, Vol. 11(3:19), pp. 1–57, 2015. [Online serial]. Available: \href{https://lmcs.episciences.org/1595}{\color{blue}{\underline{https://lmcs.episciences.org/1595}}}. [Accessed June 20, 2019]. \\
2.In this article Vogels \textit{et al.} discuss VeriFast, a tool for modular verification of safety and correctness properties of single threaded and multithreaded C and Java programs. \\
3. The authors present a formal definition and soundness proof of a core subset of the VeriFast program verification approach. \\
4. The article provides a detailed description of what VeriFast does and how it works, then introduces a simplified version of Verifast, ``Featherweight Verifast'', as well as another variant called ``Mechanised Featherweight Verifast''. \\
5. The article is useful to my project mainly for its description of what guarantees VeriFast provides. Surprisingly, these features are not obvious from the VeriFast website https://github.com/verifast/verifast or from internet searches. \\
6. The main limitation of the article is that it focuses on a subset of VeriFast. \\
7. The authors list future work to be done to create formal definitions and proofs for those features of VeriFast that are not included in Featherweight Verifast. \\
8. This article will not form the basis of my research, however it will provide useful supplementary information for comparing different software verification tools. \\

% Technical Report CW-520, Department of Computer Science, Katholieke Universiteit Leuven, Belgium, August 2008
\bibitem{theverifastprogramverifier} B. Jacobs and F. Piessens, ``The Verifast Program Verifier'', Department of Computer Science, Katholieke Universiteit Leuven, Belgium, Tech. Report. CW-520, August 2008. Available: \href{https://people.cs.kuleuven.be/~bart.jacobs/verifast/verifast.pdf}{\color{blue}{\underline{https://people.cs.kuleuven.be/~bart.jacobs/verifast/verifast.pdf}}}. [Accessed June 20, 2019]. \\
2. This report describes some aspects of an early version of the VeriFast Program Verifier.\\
3. Based on the title of this paper, I had hoped it would help me understand the advantages of using VeriFast, however this was not the case.  This report is not particularly useful for my  project, as it describes the design of an early version of the tool, and does not explain the guarantees that using the tool provides.\\
4. \\
5.  \\
6. \\
7. \\
8. \\

% The Verifast Program Verifier: A Tutorial
\bibitem{theverifastprogramverifiertutorial} B. Jacobs, J. Smans, and F. Piessens, ``The Verifast Program Verifier: A Tutorial'', Department of Computer Science, Katholieke Universiteit Leuven, Belgium, [online document], November 28, 2017. Available: \href{https://people.cs.kuleuven.be/~bart.jacobs/verifast/tutorial.pdf}{\color{blue}{\underline{https://people.cs.kuleuven.be/~bart.jacobs/verifast/tutorial.pdf}}}. [Accessed June 20, 2019]. \\
2. This report gives useful examples of how to use the VeriFast Program Verifier to prove the absence of buffer overflows, data races and functional correctness as specified by preconditions and postconditions.\\
3. \\v
4. \\
5.  \\
6. \\
7. \\
8. \\

% VCC - A practical Verification Methodology for Concurrent Programs
\bibitem{practicalverificationmethodology} E. Cohen, M. Moskal, W. Schulte, and S. Tobies, ``A Practical Verification Methodology for Concurrent Programs'', Microsoft Research, [online document], February 12, 2009. Available: \href{https://www.microsoft.com/en-us/research/wp-content/uploads/2016/02/concurrency3.pdf}{\color{blue}{\underline{https://www.microsoft.com/en-us/research/wp-content/uploads/2016/02/concurrency3.pdf}}}. [Accessed June 22, 2019]. \\
2. This document focusses on a methodology for verifying concurrent programs using VCC. It covers some theory behind the tool, but is not particularly useful for my thesis. It does not provide a list of the features the tool provides, or simple examples to demonstrate its use.\\
3. \\
4. \\
5.  \\
6. \\
7. \\
8. \\

% Verifying C Programs: A VCC Tutorial
\bibitem{vcctutorial} E. Cohen, M. Hillebrand, S. Tobies, M. Moskal, and W. Schulte, ``Verifying C Programs: A VCC Tutorial'', University of Freiburg, [online document], July 10, 2015. Available: \href{https://swt.informatik.uni-freiburg.de/teaching/SS2015/swtvl/Resources/literature/vcc-tutorial-col2.pdf}{\color{blue}{\underline{https://swt.informatik.uni-freiburg.de/teaching/SS2015/swtvl/Resources/literature/vcc-tutorial-col2.pdf}}}. [Accessed June 22, 2019]. \\
2. This tutorial gives useful examples of how to use the VCC verification tool.\\
3. \\
4. \\
5.  \\
6. The tutorial indicates that is a ``working draft, version 0.2'', and I notice that it is not available on the Microsft website, rather it is available indrectly via the University of Freiburg. In spite of this, I found it well written, and it was easy to follow the examples and perform the proofs myself.\\
7. \\
8. \\

% https://www.microsoft.com/en-us/research/project/vcc-a-verifier-for-concurrent-c/
\bibitem{vccwebsite} M. Moskal, ``VCC: A Verifier for Concurrent C'', microsoft.com, December 10, 2008. [online]. Available: \href{https://www.microsoft.com/en-us/research/project/vcc-a-verifier-for-concurrent-c/}{\color{blue}{\underline{https://www.microsoft.com/en-us/research/project/vcc-a-verifier-for-concurrent-c/}}}. [Accessed June 22, 2019]. \\
2. This is the website for the VCC verification tool.\\
3. \\
4. \\
5. \\
6. \\
7. \\
8. \\

% http://www.openjml.org/
\bibitem{openjmlwebsite} OpenJML, ``Does your program do what it is supposed to do?'', http://www.openjml.org, todo - undated. [online]. Available: \href{http://www.openjml.org/}{\color{blue}{\underline{http://www.openjml.org/}}}. [Accessed June 22, 2019]. \\
2. This is the website for the OpenJML verification tool.\\
3. \\
4. \\
5. \\
6. \\
7. \\
8. \\


\bibitem{lamport} L.~Lamport, \emph{\LaTeX: A Document Preparation
System}, 2nd ed. (Addison-Wesley, 1994).
\bibitem{LABEL2} REFERENCE 2
\bibitem{ETC.} Etc.
\end{thebibliography}

\end{document}
